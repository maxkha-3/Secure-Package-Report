\subsection{Logistics process simulator} \label{section:simulation}

To be able to test how the system reacts to different situations, a rather simple simulation application for the transport of packages was created in collaboration with Axel Vallin, who used the same simulation principal for the blockchain-based implementation of the service. It was decided to not depend on real deliveries and sensors, which could complicate and slow down the process of testing the functionality. As an additional benefit this made it possible to demonstrate the whole system in a simple and clear way.

The simulation was developed as a web page with a map and graphs, that showed the simulated data. Google Maps Directions API  was used to find a route to the delivery point, which was specified in the agreement \citep{mapsapi}. This gave an array with steps, that represented the turns, that a road-going vehicle (in case of the logistics process, a delivery truck is a relevant example) would have to make to reach the destination. To deal with the differences in lengths between turns, the estimated travel time, provided by Google Maps Directions API, for each step was used to wait at each step on the map. After one second, the total travel time was updated by a specified amount, and if that was more than the total travel time required to reach the next turn, the map was updated. At each travel time update, a data point was added to each sensor graph. Each step along the route also had a longitude and latitude position associated with it, this represented the data from the GPS.

For the rest of the sensors (temperature, acceleration, pressure and humidity), a set starting value randomly increased or decreased by a small amount during each step in the delivery. To simulate accidents like dropped packages, each sensor also had a small chance to drastically change its value.

By providing the unique agreement identifier, the simulation application is able to retrieve data from the database, that was specified by the buyer and seller (terms of the agreement), as was proposed in Section \ref{section:features}. This represented the sensors updating their configuration based on that information.

The configuration of the sensors (except the GPS) had a threshold, which made sure that the sensors only sent data to the smart contract when a violation of the agreement had occurred.