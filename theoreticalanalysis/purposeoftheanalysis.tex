\subsection{Purpose of the analysis}
One of the goals of this thesis work is to apply blockchain-specific properties to improve a system, that uses a client-server architecture. As was mentioned earlier, this can be achieved by performing a theoretical analysis of features during the design phase. In case of this study, the system, which was implemented as a result of such an analysis, is to serve the same purpose and have similar functionality to the original Blocket Secure Package with some extensions, discussed in Section \ref{section:improvementsfromoriginal}. 

This approach of using inspiration from both models of implementation is desirable, as it has a potential to address the typical drawbacks, associated with a purely centralized approach, and to improve some aspects of the system, by combining technologies. In order to make it happen, a proper evaluation of each approach is performed in a theoretical analysis. This analysis is based on theoretical assumptions and previous studies from various sources. The outcomes are used to draw a series of conclusions, regarding the drawbacks and advantages of each approach. Original system by Blocket is also taken into consideration, as it serves as a good reference point, when analyzing the qualities of traditional centralized systems. The results and conclusions of this analysis are used to compile a list of desired features for the implementation process in Sections \ref{section:features} and \ref{section:blockchainfeatures}.



