\subsection{Background}
\emph{\Gls{Blockchain}} technologies have grown to become an immensely popular sphere of information technologies. Despite being a rather young technology (at the time of writing), some of it's major advantages, turned out being quite beneficial in a selection of use cases, which allowed it to become more and more adopted by the society during past couple of years.

The most widely-known application of blockchain technology are \emph{\glspl{cryptocurrency}} like Bitcoin \citep{bitcoinwhitepaper} and \emph{\Gls{Ethereum}} \citep{buterin2013ethereum} (which is involved in this thesis). During year 2017, all major cryptocurrencies saw a drastic increase in value, which was mostly motivated by the growth of blockchain-related technology, as well as all the hype surrounding it. Many analysts are comparing the development of current blockchain related events to the burst of Dotcom Bubble \citep{dotcombubble}, that happened around the millennium shift. There is a lot of hope, speculation and money involved in blockchain technologies, as it was with the Dotcom Bubble, and there, of course, are a lot of expectations which are yet to be met. It is considered by many that blockchain has a very bright future ahead of it \citep{blockchaingreatestinvention}, but the technology has yet to prove itself being generally better, more efficient and secure, then the existing solutions.

As will be discussed later, blockchain technologies have quite a few major "selling points", such as \emph{\gls{decentralization}}, scalability, security etc, but there are, as well, quite a few non-neglectable downsides, such as high latency and lower performance, compared to regular traditional systems. Those blockchain-specific disadvantages have to be considered, when developing blockchain-based systems, however, in many cases, these downsides are simply not taken as seriously as they should be. A real world example of that is the transaction fees of major cryptocurrencies. When the largest and most widely used cryptocurrency network, Bitcoin, gets heavily congested (when transaction volume in a given timeframe drastically increases), the transaction fees to the network tend to skyrocket. On the 21st December 2017, the average network fee hit an all time high of around 37 USD per transaction \citep{btctxfee}. As Bitcoin transaction fees do not directly depend on the amount of currency being transferred, the fee is barely noticeable for wealthy individuals, financial institutions and organizations that transfer large amounts of currency per transaction. But for an average user, it would be a huge and noticeable overhead. According to statistics, made by Statista, the average amount spent per purchase transaction with Visa card in Europe from 2010 to 2015 is around 60 Euros \citep{averagevisatx} (or around 75 USD at the time of writing, for comparison). Thus, one would need to pay around 49\% fee for an average trip to the grocery store. There currently is only a tiny fraction of all payments in the world, that are made using Bitcoin network. It is scary to imagine what the transaction fees would be like if the current version of Bitcoin was to handle the same transaction density as Visa does. Clearly, the Bitcoin network has to come up with a solution for this issue before becoming the future of money, as it is referred to sometimes \citep{bitcointhefutureofmoney}. 

\pagebreak

Nowadays, many new systems are being developed using blockchain technology, rather than following the \emph{\gls{traditional model}} with usage of a central server. This approach is often chosen without a proper analysis, as the huge hype for blockchain technologies makes developers and executives forget about the obvious downsides that are associated with it. Within the IT industry, the word "blockchain" tends to be associated with something very modern, crisp and revolutionary. Some companies, that are struggling profit-wise, are choosing the approach of adding the word "blockchain" into their company name to gain publicity and media coverage. A perfect example for that is an ice tea brewing company based in the United States, called "Long Island Iced Tea Corp", that changed its name to "Long Blockchain Corp" in December 2017. This company was publicly listed at Nasdaq and this sudden company name change alone led to a 500\% spike in its stock price in the following week \citep{iceteablockchain}. One can hardly disagree with the fact that it was an overreaction by the market, which was driven by hype and speculation alone.





