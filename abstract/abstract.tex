\section*{Abstract}

Blockchain technologies have gradually gained popularity since the beginning of 2010. As of 2018, many companies and financial institutions are redesigning and building new systems with blockchain technologies as major foundation. On paper, the blockchain has numerous advantages over the traditional centralized approach, however, this study showed, that there are some large drawbacks, which are associated with usage of blockchain. The most significant downsides are blockchain's low performance, enormous cost and high environmental impact, compared to traditional client-server based systems. 

Therefore, the overall goal of this study was to highlight the importance of considering these drawbacks and discuss, how performing of a detailed feature analysis during the design phase, might guide application developers to the correct path, during the implementation phase of a system, when blockchain is considered being an alternative to the traditional client-server approach. As the result of this study, it turned out, that both client-server and blockchain based approaches do have their respective use cases and disadvantages. A conclusion was drawn, that the best approach would be either to use a mix of both technologies, or to use the blockchain as a verification mechanism behind a client-server backend, in order to improve its data integrity and persistence quality attributes.

\pagebreak

\section*{Acknowledgements}

I would like to thank Luleå University of Technology for providing me with the knowledge and skills, which I gathered during my five years of studies. Those skills were necessary in order to make this thesis work possible. Big shout out to my supervisor at LTU, Olov Schelén, for providing me with the idea of performing a comparison-based study, when my initial thesis work proposal was rejected, and giving me necessary feedback and advice during the course of this project. Huge thanks to my opponent, Tobias Axelsson, for compiling the incredibly detailed list of potential improvements to the report and giving his opinion on the context of this study. I would also like to thank Data Ductus for giving me an opportunity to perform my thesis work at the supervision of Mario Toffia, who was very helpful and was available whenever a piece of advice was needed. I am also thankful for an opportunity of going abroad for an exchange semester at Clarkson University during Fall 2017. I developed my English a lot during my time in the United States and learned a lot about automatons, state machines and handling of big datasets. Those skills were very useful for this thesis work. I also met some wonderful people during my exchange semester, one of which, Guillaume Lesot, helped me with preparations for the presentation and provided me with a feedback about the report on multiple occasions. Special thanks goes to Axel Vallin, whom I have been studying with since upper-secondary school and who was working on another part of this study as his own separate thesis work. We helped eachother a lot along the way and we have been through a lot. And, of course, most importantly, I would like to thank my family, my beloved fiancée Anzhelika and close friends of mine for all the support. You are my everything.
\newline
\newline

\hspace{8.5cm} Maxim Khamrakulov, June 2018
