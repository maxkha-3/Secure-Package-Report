\subsection{User stories} \label{section:userstories}

In order to describe different use cases for the centralized Secure Package application, a number of user stories were considered. The user stories consider different scenarios, in which the choice of using the centralized Secure Package application may potentially solve some problems.

\subsubsection{Basic agreement}
Thomas wants to sell a pair of gloves via an advertisement on an online marketplace. Since he does not trust strangers, he wants to make sure that he is not tricked by the buyer. He also finds it difficult to trust a third party, as he has had bad experience with large companies fooling him in the past.

Even though the centralized Secure Package application involves a degree of third party, it provides an additional degree of privacy and the explorer functionality introduces a tracing option, since all the events are recorded and made public. Thomas therefore decides to use the centralized service.

Hasse, who is interested in the gloves, but has no idea of who Thomas is, likes this service in comparison to the original Secure Package. After he and Thomas have agreed on the deal, he sends the payment to the blockchain. When the gloves arrive and Hasse is satisfied, the payment is automatically transferred to Thomas.

\subsubsection{Trade of valuable items}
After having experienced a successful agreement carried out by the service, Hasse decides to use it again a few months later when he is going to sell half of his very valuable stamp collection. Since he is worried that water might damage the stamps, he wants to add a humidity sensor to the package.

Hasse includes the sensor information in the agreement and sends the package to Kajsa who wants to add them to her own collection. By including the sensor data in the contract, Hasse and Kajsa can be certain than no one tampers with it afterwards to cover something up. Kajsa receives the package and can see that no violation regarding humidity has occurred.

\subsubsection{Damaged goods}
When Hasse later decides to give up on the stamps altogether, Kajsa buys the rest of his collection. They decide on using the same service and add a sensor exactly as before. This time, Kajsa is not happy with the condition of the stamps and sends them back.

Hasse, as he receives the returned package, also notices that the stamps are damaged. He can then use the application's agreement dashboard in order to check, whether the sensors have detected a violation of humidity threshold, which turned out to be the case. Hasse then initiates a conflict request, which is submitted to a clerk administrator. Clerk notices that a violation of terms has happened during the transport and the sensor data confirms that water has come into contact with the stamps. The clerk decides that the logistics company is responsible for the damage and should provide a compensation to Hasse. When the logistics company provides compensation for the damaged goods, both Hasse and Kajsa will get their money from the service.

\subsubsection{Fiat payment option} \label{section:fiatoption}
Viktor would like to sell his phone, using an online marketplace. He had previous experience with using a blockchain-based system, similar to the centralized Secure Package. This system used Ether for payments, which is extremely volatile. Previously, he sold a headset for a price of 0.5 ETH, using the blockchain-based application. At the time of shipping the price of Ether was at around 700 USD. Once the headset was received by the buyer, the Ether price had suddenly dropped to 450 USD. Thus, instead of receiving an equivalent of 350 USD, Viktor received only 225 USD, which made him very upset.

Viktor therefore decides to use the centralized Secure Package, as it supports payments with fiat money, which is alot less volatile.