\subsubsection{Economic aspect} \label{section:economics}
Economic aspect is very important to take into consideration, as it is essential, when designing centralized applications, especially, when there is a central authority involved. Blockchain-based systems often rely digital assets in form of cryptocurrencies, which are extremely volatile. An introduction of sensor support opens up opportunity for logistics companies to make profit from adding sensors to packages.

\paragraph{Price volatility of cryptocurrencies}
The nature of Secure Package service is, whether it is Ethereum network or a centralized server architecture that is used for data handling, the logistics process stays the same, apart from usage of sensors for monitoring. The time it takes for the item to reach the buyer is measured in days, which poses a problem when using a cryptocurrency-based payment system. The reason for that is that cryptocurrencies tend to be extremely volatile from economic standpoint.

In order to demonstrate application of that in real world, consider a case, in which buyer $A$ agreed to purchase an item from seller $B$ using the Secure Package service, that was built using blockchain architecture, for the price of 1 ETH back in 27th of May 2017. Assuming that $B$ sent the item on the following day and that shipment process took two days, the package was received and its condition was confirmed by $A$ on 30th of May. Thus, it took the total of three days for the seller to receive the payment from the point of agreement. Currently, vast majority of people are measuring the value of items in fiat money. When trading goods in Ether, people would naturally think of fiat value and then convert it to Ether. On the 27th of May 2017, the price of ETH had its low of around 130 USD. On the afternoon of 30th of May, the price jumped all the way up to 215 USD, which is an increase of around 65\% over the course of only three days \citep{ethprice}. Thus, the buyer had to pay 65\% more for the item in fiat currency equivalent.

Usage of fiat money for payments is a lot less volatile, however it is not possible to integrate such fiat payment system into a blockchain. The closest possible solution for that is usage of, so called, "stable coins", which are covered in the next paragraph.

\paragraph{Stable coins and associated issues}
There are currencies, such as Tether \citep{tether}, which are often referred to as "stable coins". The ideology behind those currencies is that they are pegged to the value of a given fiat currency one-to-one. In Tether, this one-to-one exchange rate is achieved by maintaining a US dollar reserve. Each issued Tether is corresponding to one US Dollar in the reserve, maintained by a company, called Tether Limited. This results in Tether being worth approximately one US dollar per token. An integration of such "stable coin" functionality could potentially solve the issue of volatility, by creating an ERC20 token with a money reserve. However, this requires a third party, which maintains the reserve fund, thus making the whole token system centralized. The third party presence and its potential disadvantages were discussed in \ref{section:analysisthirdparty}.

\paragraph{Introduction of sensor support}
Many companies are using marketing strategies, in which they charge customers more money for additional features in a product. An introduction of sensor support for logistics process may not only provide additional security for the packages, but can also benefit the logistics company economically. Implementation of sensors, such as accelerometers, temperature and humidity indicators, into the logistics process would open up a possibility of generating more revenue from package delivery, by charging extra money for inclusion of sensors.

\paragraph{Gas usage and transaction density}
Each and every interaction that is processed by Ethereum network requires a transaction fee. This fee is measured in specific units, called gas in Ethereum and it is heavily dependent on current congestion of the network (high congestion results in higher gas price required to process a transaction within a given time frame). At the time of writing, the average plain transaction that does not execute any specific smart contract code consumes 21000 gas units, which costs around 0.00004 ETH (which is around 0.03 USD$^\star$) to process, given the current standard gas price of 2 Gwei \citep{ethgasstation}. However, computations, that are executed by the smart contracts consume a fixed amount of gas per computation. Thus, method calls, that require large amounts of computations, consume more gas. This leads to a conclusion, that a lot of time and effort needs to be dedicated to reduce function calls and optimize executable calls in smart contracts in order to avoid large transaction fees.

Introduction of sensor support, especially GPS tracking, introduces a potential problem of bottlenecking communication bandwidth of sensor data. The reason for that does not lie in the technical aspect of the system. As mentioned before, every interaction with smart contracts consumes gas. All of the information about the agreement, including GPS sensor data, is stored inside the smart contract. Thus, to update the GPS data, an interaction with the smart contract, that updates the latitude and longitude values, is required. Execution of variable update operation $G_{sreset}$ has a base gas consumption of 5000. Thus, the amount of gas, which is consumed by updating two variables inside the smart contract, is equal to at least 10000. Apart from execution of code, the transaction itself has a base gas consumption of 21000, whether it is a simple transfer of tokens from one address to another, or an execution of some advanced algorithm on a smart contract, that requires a lot of computational resources. Thus, continuous communication with sensors during the logistics process would lead to a large number of generated transactions, that consume at least 31000 gas. This is not really a problem, when dealing with sensors like accelerometers, where the only relevant data is whether the threshold value was violated. This data can be sent to the contract once the logistics process is finished, thus only generating one transaction. GPS tracker, however, becomes quite useless, unless it communicates with the application somewhat often.

In order to get an understanding of the scope of this issue, the following scenario is considered. Assuming that the logistics process takes exactly two days to complete and that the included GPS sensor communicates with the contract once a minute, simple math reveals that 2880 transactions are generated during that timeframe. Each transaction consumes 31000 gas (21000 for performing the transaction and around 10000 for performing the update of each variable, depending on the size of the variables), which results in $8.928 \cdot 10^7$ units of gas consumed in the process. Now, when dealing with system design, it is important to consider the worst possible cases. As mentioned before, the gas price needs to be increased, when network becomes more congested, otherwise, the transaction will not be mined until the congestion decreases, which can take days. Historical data shows that the average gas price was approximately 9 Gwei on the 10th of January 2018. That big of a gas price was resulted from a large amount of transactions being processed around that day. Considering the gas price of 9 Gwei, each sensor communication would come at a cost of 0.00028 ETH. This sums up to approximately 0.81 ETH during two day transfer, which is around 567 USD$^\star$. It is an understatement to say that this is a large amount of money to spend for this kind of functionality, unless the contents of the package are of high value. In order to avoid that problem, the frequency of sensor communication needs to be significantly reduced, which indirectly introduces a bottleneck.

\paragraph{Gas usage and storage}
Apart from economic gas related issues, regarding increased transaction density, there are also issues associated with storage on Ethereum blockchain. As mentioned in Section \ref{section:scalability}, each smart contract has $2^{261}$ bits of individual virtual storage, which is a huge amount. However, as was discovered in Section \ref{section:scalability}, this storage is extremely expensive. Each {\small SSTORE} operation, that stores a 256-bit word into the smart contract's empty space consumes 20000 gas.

When selling items online, a good picture and detailed description increases the item's selling potential dramatically \citep{ebay}, thus it is essential and very important to support inclusions of pictures and descriptions in the adverts, when creating a merchandise trading platform. Table \ref{tab:imgdescscenarios} illustrates different gas consumption scenarios, in which image and description upload is performed. The average image size is assumed being 100 kilobytes (after compression, which is assumed being done off-chain before the upload). 

Table \ref{tab:imgdescscenarios} shows that it costs a total of 183.90 USD$^\star$ to post an advert, that consists out of two heavily compressed images and a short description. By analyzing the table, it is easy to spot that additional images are extremely expensive to upload. An important note is that 100 kilobytes per image is a very generous approximation, as it is not uncommon for a typical 1024x768 JPEG image to have a size of around 1 megabyte, which is 10 times larger, than was considered in the scenarios. Greater image sizes and longer descriptions would be proportionally more expensive to upload, as derived in Equation (\ref{eq:newstorecost}). It is quite obvious, that storage of large files in Ethereum is not viable due to the high cost.

\begin{table}[H]
\centering
\begin{tabular}{| c | c | c | c |}
\hline
\textbf{Advertisement scenario}& \textbf{Gas used} & \textbf{ETH} & \textbf{USD$^\star$}\\
\hline 
0 images, 100 characters description & $1.3 \cdot 10^5$ & 0.0002592 & 0.18\\
0 images, 200 characters description & $2.0 \cdot 10^5$ & 0.0003993 & 0.28\\
0 images, 400 characters description & $3.3 \cdot 10^5$ & 0.0006614 & 0.46\\
1 image, 200 characters description & $6.6 \cdot 10^7$ & 0.1316073 & 92.13\\
2 images, 100 characters description & $1.3 \cdot 10^8$ & 0.2627172 & 183.90\\
\hline
\end{tabular}
\caption{Different uploading scenarios. Gas price of 2 Gwei was considered. Values within the table were derived, using Equation \textnormal{(\ref{eq:newstorecost})}.}
\label{tab:imgdescscenarios}
\end{table}

\paragraph{Removal of third party}
Introduction of a blockchain architecture has a potential to remove the third party aspect, in form of Blocket, from the equation. It is obvious that companies, that are maintaining a service, need to get some sort of payment in return. In case of Blocket Secure Package, dedicated servers, support staff and maintenance technicians are needed to keep the service operational. Blocket is a profit seeking organization, thus it is not hard to guess that the service would become cheaper if the third party involvement was removed. And so is the case, as it costs around 3 USD less to send a small package directly though DBSchenker, then by using Blocket Secure Package. However, what a user does not get, by just sending the package directly through DBSchenker, is the cash-on-delivery service and an opportunity of contacting Blocket's customer service if something goes wrong with the transfer.
  




