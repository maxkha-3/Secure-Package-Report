\subsubsection{Third party involvement and trust} \label{section:analysisthirdparty}
Trust is a sensitive subject in context of this study. Theoretically, usage of a decentralized blockchain, like Ethereum, is very much desirable from this point of view, as the third party involvement is completely eliminated by doing that. Centralized services, that are built using client-server architecture, are, in vast majority of cases, being maintained by some kind of third party organization. In this context, the term "third party" means an entity that is indirectly involved in an interaction between a set of users (typically a central authority). Having a third party that maintains control over the system, leads to all of the data being exposed to an indirectly involved entity, which in its turn implies that this entity has full control over the data (in most cases). This can be a potential source of a wide range of problems.

\paragraph{Identity frauds and data leaks}
Many of today's systems require users to provide their sensitive private information (such as their name, bank account number, address, job occupation, etc.), in order to function. This means that all of the data ends up being stored on the servers, which the user is not in control of. As the third party, that runs the service, has full control of these servers, as well as all of the data that is being stored on them, there is technically nothing that prevents the third party from misusing and exploiting that data for own purposes. Identity thefts are a very common type of crime nowadays. Once the criminals gain access to the sensitive private information, there is not much that can stop them from, for example, taking out a loan in victim's name, which can result in this victim being responsible for paying the loan back. According to Insurance Information Institute, the 2017 Identity Fraud Study found that 16 billion USD was stolen from 15.4 million U.S. consumers in 2016 \citep{idthefts}.

During the past decade, data exchange services like Dawex and BDEX has grown immensely. Anyone can register on these services and put a dataset up for sale. The big data exchange services have a policy of blocking all datasets that contain sensitive private information, but this can be avoided by encrypting the data, or by selling it through smaller services that does not have that policy. In theory, any company that uses a centralized ledger may extract all of the personal data and put it up for sale, by using an alias, to not disguise themselves.

\paragraph{Money frauds}
Some services have functionality of holding some of the user's money (like online casinos and such). In that case, users hand their money over to the third party, which maintains full control of those funds. If, for example, such company files for bankruptcy, users might find themselves left empty handed.

The original Blocket Secure Package fits into this category of services, as the money is stored in Blocket's account, during the merchandise transfer process. One can never be sure what Blocket does with the money until it is transferred to the seller. And, as mentioned before, there is also a theoretical possibility of Blocket seizing their operations and disappearing with the user's money.

\paragraph{Reputation of organizations and data protection laws}
Blocket is widely used by many people in Sweden, it has good reviews, which provides potential users with a belief, that the risk of getting scammed by Blocket is relatively low. However, there is a difference in considering that something is safe and something actually being safe. Regular users rarely get to know how their data is handled. In case of Blocket, there is some information to be found on their website about the guidelines, regarding handling of user data \citep{blocketuserdata}. However, there is no real 100\% guarantee that these guidelines are followed, as users do not have a possibility to check it in practice. In other words, users have to trust the company to follow the guidelines, regarding both handling of private data and assets. 

There are strict laws, regarding protection of personal data in developed and digitalized countries, such as Sweden, and companies are required to obey them. Regular checks and audits are performed to check if the companies are following the guidelines and terms of conditions. Thus, the terms of conditions, regarding the data handling on Blocket's website can be considered rather trustworthy. However, there are other less developed countries, where no such laws exist. Companies could then propose terms of conditions, just to trick users into believing that their data is safe and do whatever they want with it, without any risk of being punished. 

\paragraph{Fiat money}
Just about all public money these days is fiat money. Fiat money is typically issued by the government of a country. The government points to something and declares it being a bearer of monetary value. Might be a coin, might be a note, but the only reason it has value is because the government says so and makes people believe them. In that case, the government acts as a third party that is issuing the monetary assets, which can lead to some serious problems, as it did in Venezuela and Zimbabwe recently \citep{venezimbab}. This thesis work's topic is not about global economics, thus the reasons for potential problems are not discussed in detail. However, it is important to consider that any fiat currency has strong bonds to the issuing contry's politics. Bad political decisions may, and in fact will, reflect on the value of the currency, making it very volatile. Usage of digital asset token payment system, as proposed in the decentralized version of the service with no central governing body, eliminates that source of potential problems.