\subsubsection{Environmental aspect} \label{analysis:environmental}

There are differences between the architecture models, regarding the environmental aspect. Usage of current version of Ethereum poses enormous environmental issues, as execution of proof-of-work algorithm is very power consuming. Servers and their cooling solutions are getting more efficient and sophisticated nowadays, thus making the model of using a centralized server more power efficient and environmentally friendly.

\paragraph{Mining and power consumption}
Execution of proof-of-work algorithms (ofter referred to as mining) is a very power hungry process. At the time of writing, the total hashrate of Ethereum network is fluctuating around 260 TH/s and is steadily increasing \citep{ethhashrate}. The most efficient hardware for mining most ASIC-resistant cryptocurrencies are graphic cards with chips from manufacturers NVidia and AMD. In order to get an idea of Ethereum network's total power consumption, let's assume that all of the miners are using the most common and widely used graphics card for this purpose, AMD RX580, which has a hashrate of around 30.2 MH/s and consumes around 135 Watts in the process \citep{whattomine}. Simple math shows, that the total hashrate of Ethereum corresponds to hashpower of around 8.6 million RX580s, which draw a whopping 1.16 TW of power combined! According to The World Bank, this figure corresponds to power consumption of the entire country of Lithuania \citep{worldpowerconsumption}. Clearly, from environmental standpoint, this is a big issue. The situation is getting worse, as mining becomes more and more popular, due to the activity of running mining hardware being profitable from economic standpoint. An illustration of that increased popularity can be seen by studying recent development of Ethereum's hashrate. By looking at the graph in Figure \ref{fig:ethereumhashrate} of Section \ref{section:analysisencryption}, we can see, that the total hashrate of Ethereum network was doubled between December 2017 and March 2018, which is a period of 4 months.

Security of public blockchain networks, such as Ethereum, is strongly dependent on the total hashpower, due to increased protection against 51\% attacks, as described in \ref{section:analysisencryption}. It is possible to run an instance of Ethereum network with only a small fraction of today's hashrate, however, this would make the blockchain fragile against 51\% attacks.

The bottom line is that in order to run any secure blockchain network that uses proof-of-work consensus mechanism, it has to have a significant number of miners. Each piece of hardware, that is used for verification purposes consumes power. Having a large number of mining hardware constantly running is bad for the environment and the situating is getting worse, with the hashrate increasing every day that comes.

\paragraph{Proof-of-stake}
Not all blockchain networks are using proof-of-work as consensus mechanism. Another mechanism, which currently is gaining popularity and recognition, is called \emph{\gls{proof-of-stake}}. Its principal is that the nodes are locking a number of assets (in case of cryptocurrencies it is tokens, or coins) into validator nodes to secure the network. In proof-of-stake based blockchains, the creator of the next block is chosen via various combinations of random selections. The process of locking the assets into the network is called staking. Larger stakes results in higher chance of finding the next block. There are various potential issues, regarding this mechanism. One of them is that the nodes with large amount of digital assets might take over the network by getting the vast majority of block rewards, thus making the blockchain less decentralized. However, this mechanism does not consume nearly enough computational power, as there is no need for advanced hardware that consumes a lot of electricity. 

One of the most well known proof-of-stake networks is Nxt \citep{nxtwhitepaper}. As there is no need for large amounts of complex number crunching, the distributed consensus mechanism can be executed by small power efficient machines, like Raspberry Pi, that consumes around 6 Watts of power. A paper, released by Matthew Czarnek in 2014, stated that if Nxt network was to reach the size of Bitcoin network, it was to draw around 8000 times less power \citep{nxtpos}. It is an understatement to say that this is a significant improvement.

Ethereum is in a process of moving from a proof-of-work consensus mechanism to a partial proof-of-stake. This future release of Ethereum is called Casper, however the exact details regarding that have not yet been revealed at the time of writing \citep{ethpos}. 

\paragraph{Central server architecture and its power efficiency}
In order to grasp how the usage of a centralized architecture measures against proof-of-work based networks like Ethereum, in terms of power efficiency, consider the largest social network in the world Facebook. During year 2016 all of the Facebook's servers combined used a total of 1830 GWh of electricity \citep{facebookserverpower}, which converts to power draw of around 210 GW. As described in the discussion about the power draw associated with mining, the total power draw by Ethereum miners is approximately 1.16 TW. This figure is roughly 5 to 6 times bigger, and this is when such a large client-server application service like Facebook is considered. The goal of this analysis is to apply the concepts to a rather small-scaled merchandise trading platform, which requires significantly less amount of server units than Facebook, hence a lot less of a power draw and environmental impact. 

\paragraph{Repetitive code execution in Ethereum}
As mentioned earlier, mining itself consumes a lot of power. There is a large number of DApps, that are deployed on Ethereum. It is very important to consider that the consensus mechanism is distributed across the mining nodes. One part of the mechanism is the verification of executable code. This leads to the same smart contract code being randomly executed multiple times, over and over again, on multiple nodes. This can obviously be considered as a waste of computational power, but it is necessary for verification process. Thus, every DApp and smart contract, that is deployed on Ethereum network, is being executed on multiple nodes and requires multiple units of computational power per instruction. This can be compared to a centralized architecture, where relevant code is executed exactly once, unless an error or a bug is encountered. Thus, the execution of the same set of instruction results in more computational effort for source code, deployed on Ethereum.

